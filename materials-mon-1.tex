\documentclass{article}
\usepackage{booktabs}
\usepackage{geometry}
\usepackage{hyperref}
\usepackage{indentfirst}
 \geometry{
 a4paper,
 total={170mm,257mm},
 left=20mm,
 top=20mm,
 }
\usepackage[utf8]{vietnam}
\hypersetup{
    colorlinks=true,
    linkcolor=blue,
    filecolor=magenta,      
    urlcolor=cyan,
    pdftitle={Overleaf Example},
    pdfpagemode=FullScreen,
    }

\urlstyle{same}
\title{Session 3: Logical Mathematics and Literature}
\author{Linh}
\date{10/04/2023}
\begin{document}
    \maketitle
    \section{Logical Mathematics}
    \subsection{Theory}
    \begin{itemize}
        \item Remember the definitions of negation, conjunction, disjunction, Exclusive OR, implication.
        \item Can rewrite a langual statement into a logical mathematics statement.
        \item Can write truth table
        \end{itemize}
    \subsection{Exercises}
    \begin{enumerate}
        \item Which of these are propositions? What are the truth values of those that are propositions?
            \begin{enumerate}
                \item Do not pass go.
                \item What time is it?
                \item There are no black flies in Maine.
                \item $4+x=5$.
                \item The moon is made of green cheese.
                \item $2^n \geq 100$. 
            \end{enumerate}
        \item What is the negation of each of these propositions?
            \begin{enumerate}
                \item Mei has an MP3 player.
                \item There is no pollution in New Jersey.
                \item $2 +1 = 3$.
                \item $5+ 3 = 6$.
                \item The summer in Maine is hot and sunny.
            \end{enumerate}
        \item Suppose that during the most recent fiscal year, the annual revenue of Acme Computer was 138 billion dollars and its net profit was 8 billion dollars, the annual revenue of Nadir Software was 87 billion dollars and its net profit was 5 billion dollars, and the annual revenue of Quixote Media was 111 billion dollars and its net profit was 13 billion dollars. Determine the truth value of each of these propositions for the most recent fiscal year.
            \begin{enumerate}
                \item Quixote Media had the largest annual revenue.
                \item Nadir Software had the lowest net profit and Acme Computer had the largest annual revenue.
                \item Acme Computer had the largest net profit or Quixote Media had the largest net profit.
                \item If Quixote Media had the smallest net profit, then Acme Computer had the largest annual revenue.
                \item Nadir Software had the smallest net profit if and only if Acme Computer had the largest annual revenue.
            \end{enumerate}
        \item Let \textit{p} and \textit{q} be the propositions "The election is decided" and "The votes have been counted," respectively. Express each of these compound propositions as an English sentence.
            \begin{enumerate}
                \item $\neg p$
                \item $p \vee q$
                \item $\neg p \wedge q$
                \item $q \rightarrow p$
                \item $\neg q \rightarrow \neg p$
                \item $\neg p \rightarrow \neg q$
                \item $\neg q \vee (\neg p \wedge q)$
            \end{enumerate}
        \item Let \textit{p, q}, and \textit{r} be the propositions
            \begin{itemize}
                \item \textit{p}: You get an A on the final exam.
                \item \textit{q}: You do every exercise in this book.
                \item \textit{r}: You get an A in this class.
            \end{itemize}

            Write these propositions using \textit{p, q}, and \textit{r} and logical connectives (including negations).
            \begin{enumerate}
                \item You get an A in this class, but you do not do every exercise in this book.
                \item You get an A on the final, you do every exercise in this book, and you get an A in this class.
                \item To get an A in this class, it is necessary for you to get an A on the final.
                \item You get an A on the final, but you don't do every exercise in this book; nevertheless, you get an A in this class.
                \item Getting an A on the final and doing every exercise in this book is sufficient for getting an A in this class.
                \item You will get an A in this class if and only if you either do every exercise in this book or you get an A on the final. 
            \end{enumerate}
        \item Construct a truth table for each of these compound propositions.
            \begin{enumerate}
                \item $p \rightarrow \neg p $
                \item $p \oplus (p \vee q) $
                \item $(p \wedge q) \rightarrow (p \vee q)$
                \item $p \rightarrow (\neg q \vee r)$
                \item $(p \oplus q) \vee (p \oplus \neg q)$
                \item $(p \rightarrow q) \rightarrow (q \rightarrow p) $
                \item $\neg p \oplus \neg q$
                \item $\neg p \rightarrow (q \rightarrow r)$
                \item $(p \rightarrow q) \vee (\neg p \rightarrow r)$
                \item $(p \rightarrow q) \wedge (\neg p \rightarrow r)$
            \end{enumerate}
        \item Explain, without using a truth table, why $(p \vee \neg q) \wedge (q \vee \neg r) \wedge (r \vee \neg p)$ is true when \textit{p, q}, and \textit{r} have the same truth value and it is false otherwise.
    \end{enumerate}
    \section{Literature}
    Đọc "Những chiếc ấm đất" của Nguyễn Tuân: \texttt{https://trabavan.com/nhung-chiec-am-dat-nguyen-tuan/}
\end{document}