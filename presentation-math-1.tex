\documentclass{beamer}
\usetheme{Copenhagen}
\usepackage[utf8]{vietnam}

\title{POLYGON Project: Session 3}
\subtitle{Mathematics 1 - Logic and Proofs}
\author{Linh}
\date{2023-04-08}

\begin{document}

\frame{\titlepage}
\begin{frame}
    \frametitle{Table of Contents}
    \tableofcontents
    \end{frame}
    \AtBeginSubsection[]
{
  \begin{frame}
    \frametitle{Table of Contents}
    \tableofcontents[currentsection]
  \end{frame}
}
\section{Introduction}
\begin{frame}
\frametitle{Introduction}

\begin{definition}{Logic}
    \textit{is the basis of all mathematical reasoning.}
    \end{definition}

\end{frame}
\begin{frame}
    \frametitle{How to understand mathematics?}
    \begin{itemize}
        \item We must understand \textbf{what makes up} a correct mathematical argument. That is \textbf{proof} (chứng minh).

        \item We prove a mathematical statement is true $\longrightarrow$ \textbf{theorem} (định lí).

        \item To learn a mathematica topic, you need to \textbf{construct} mathematical arguments on this topic, not just read exposition.
    \end{itemize}
    \end{frame}
\begin{frame}
    \frametitle{What you should expect after this session?}
        \begin{itemize}
            \item Explain \textbf{what makes up} a correct mathematical argument.
            \item Introduce \textbf{tools} to construct these arguments.
        \end{itemize}
    \end{frame}
\section{Propositional Logic (Logic mệnh đề)}    
\begin{frame}
    \frametitle{Propositions (Mệnh đề)}
    \begin{definition}{Proposition}
        \textit{is a declarative sentence (fact) that is either true or false, but not both.}
        \end{definition}
    \end{frame}
    \begin{frame}
        \frametitle{Propositions (Mệnh đề)}
        \begin{itemize}
            \item Denote: use letters like \textit{p, q, r, s, t,...} to denote propositions.
            \item \textbf{Truth value} of a proposition: \textit{true} denoted by T or \textit{false} denoted by F.
            \item \textbf{Compound propositions} (mệnh đề phức hợp) are formed by combining propositions using \textbf{logical operators} (toán tử logic).
        \end{itemize}
        \end{frame}
\subsection{Negation (Phủ định)}
\begin{frame}
    \frametitle{Negation (Phủ định)}
        Let \textit{p} be a proposition. The \textbf{negation} of \textit{p} ($\neg p$ - not p), is the statement: \textit{"It is not the case that p"}.

        The truth value of $\neg p$ is the opposite of the truth value of \textit{p}.
    \end{frame}
    \begin{frame}
        \frametitle{Truth Table (Bảng chân trị)}
        \begin{center}
            \begin{tabular}{|c|c|}
                \hline
                p & $\neg p$\\
                \hline
                T & F \\
                \hline
                F & T \\
                \hline
            \end{tabular}
            \end{center}
        \end{frame}
        \begin{frame}
            \frametitle{Ví dụ}

            \end{frame}
\subsection{Conjunction (Kết hợp, phép hội)}
\begin{frame}
    \frametitle{Conjunction (Kết hợp)}
    Let \textit{p} and \textit{q} be propositions. The \textbf{conjunction} of \textit{p} and \textit{q} ($p \wedge q$ - p and q) is the proposition "p and q". The conjunction $p \wedge q$ is true when both p and q are true and is false otherwise.
    \end{frame}
\begin{frame}
    \frametitle{Truth Table (Bảng chân trị)}
    \begin{center}
        \begin{tabular}{|c|c|c|}
            \hline
            p & q & $p \wedge q$\\
            \hline
            T & T & T \\
            \hline
            T & F & F \\
            \hline
            F & T & F \\
            \hline
            F & F & F \\
            \hline
        \end{tabular}
        \end{center}
\end{frame}
\begin{frame}
    \frametitle{Ví dụ}

    \end{frame}
\subsection{Disjunction (Phân hợp, phép tuyển)}
\begin{frame}
    \frametitle{Disjunction (Phân hợp)}
    Let \textit{p} and \textit{q} be propositions. The \textbf{disjunction} of \textit{p} and \textit{q} ($p \vee q$ - p or q) is the proposition "p or q". The disjunction $p \vee q$ is false when both p and q are false and is true otherwise.
\end{frame}
\begin{frame}
    \frametitle{Truth Table (Bảng chân trị)}
    \begin{center}
        \begin{tabular}{|c|c|c|}
            \hline
            p & q & $p \vee q$\\
            \hline
            T & T & T \\
            \hline
            T & F & T \\
            \hline
            F & T & T \\
            \hline
            F & F & F \\
            \hline
        \end{tabular}
        \end{center}
        \end{frame}
\begin{frame}
    \frametitle{Ví dụ}
\end{frame}
\subsection{Exclusive OR - XOR}
\begin{frame}
    \frametitle{Exclusive OR - XOR}
    Let \textit{p} and \textit{q} be propositions. The \textbf{exclusive OR} of \textit{p} and \textit{q} ($p \oplus q$ - p xor q) is the proposition "p xor q". The exclusive OR $p \oplus q$ is true when exactly one of p and q is true and is false otherwise.
\end{frame}
\begin{frame}
    \frametitle{Truth Table (Bảng chân trị)}
    \begin{center}
        \begin{tabular}{|c|c|c|}
            \hline
            p & q & $p \oplus q$\\
            \hline
            T & T & F \\
            \hline
            T & F & T \\
            \hline
            F & T & T \\
            \hline
            F & F & F \\
            \hline
        \end{tabular}
        \end{center}
        \end{frame}
\begin{frame}
    \frametitle{Ví dụ}
\end{frame}
\subsection{Implication (Mệnh đề kéo theo)}
\begin{frame}
    \frametitle{Implication (Mệnh đề kéo theo)}
    Let \textit{p} and \textit{q} be propositions. The \textbf{conditional statement} of \textit{p} and \textit{q} ($p \rightarrow q$ - p implies q) is the proposition "p implies q". The conditional statement $p \rightarrow q$ í false when p is true and q is false, and true otherwise.

    \textit{p} is called \textbf{hypothesis}. \textit{q} is called \textbf{conclusion}. 
\end{frame}
\begin{frame}
    \frametitle{Truth Table (Bảng chân trị)}
    \begin{center}
        \begin{tabular}{|c|c|c|}
            \hline
            p & q & $p \rightarrow q$\\
            \hline
            T & T & T \\
            \hline
            T & F & F \\
            \hline
            F & T & T \\
            \hline
            F & F & T \\
            \hline
        \end{tabular}
        \end{center}
        \end{frame}
\begin{frame}
    \frametitle{Ways to express}
    $$ p \rightarrow q$$
    \begin{itemize}
        \item "if p, then q"
        \item "if p, q"
        \item "p is sufficient for q"
        \item "q if p"
        \item "q when p"
        \item "a necessary condition for p is q"
        \item "q unless $\neg p$
        \end{itemize}
    \end{frame}
    \begin{frame}
        \frametitle{Ways to express}
        \begin{itemize}
            \item "p implies q"
            \item "p only if q"
            \item "a sufficient condition for q is p"
            \item "q whenever p"
            \item "q is necessary for p"
            \item "q follows from p"
            \item "q provided that p"
            \end{itemize}
        \end{frame}
\section{Exercise}
\begin{frame}
    \frametitle{Exercise 1}
    Suppose that Smartphone A has 256MB RAM and 32 GB ROM, and the resolution of its camera is 8 MP; Smartphone B has 288 MB RAM and 64 GB ROM, and the resolution of its camera is 4 MP; and Smartphone C has 128 MB RAM and 32 GB ROM, and the resolution of its camera is 5 MP. Determine the truth value of each of these propositions.
\end{frame}
\begin{frame}
    \frametitle{Exercise 1}
    \begin{enumerate}
        \item Smartphone B has the most RAM of these three smartphones.
        \item Smartphone C has more ROM or a higher resolution camera than Smartphone B.
        \item Smartphone B has more RAM, more ROM, and a higher resolution camera than Smartphone A.
        \item If Smartphone B has more RAM and more ROM than Smartphone C, then it also has a higher resolution camera.
        \item Smartphone A has more RAM than Smartphone B if and only if Smartphone B has more RAM than Smartphone A.
    \end{enumerate}
\end{frame}
\begin{frame}
    \frametitle{Exercise 2}
    Let \textit{p} and \textit{q} be the propositions.
    \begin{itemize}
        \item \textit{p}: You drive over 65 miles per hour.
        \item \textit{q}: You get a speeding ticket.
        \end{itemize}

        Write these propositions using \textit{p} and \textit{q} and logical connectives (including negations).
\end{frame}
\begin{frame}
    \frametitle{Exercise 2}
    \begin{enumerate}
        \item You do not driver over 65 miles per hour.
        \item You drive over 65 miles per hour, but you do not get a speeding ticket.
        \item You will get a speeding ticket if you drive over 65 miles per hour.
        \item If you do not driver over 65 miles per hour, then you will not get a speeding ticket.
        \item Driving over 65 miles per hour is sufficient for getting a speeding ticket.
        \item You get a speeding ticket, but you do not drive over 65 miles per hour.
        \item Whenever you get a speeding ticket, you are driving over 65 miles per hour.
    \end{enumerate}
\end{frame}
\begin{frame}
    \frametitle{Exercise 3}
    Determine whether each of these conditional statements is true or false.
    \begin{enumerate}
        \item If $1+1=2$, then $2=2=5$.
        \item If $1+1 =3$, then $2+2=4$.
        \item If $1+1 = 3$, then $2+2 =5$.
        \item If monkeys can fly, then $1+1 =3$.
    \end{enumerate}
\end{frame}
\begin{frame}
    \frametitle{Exercise 4}
    Construct a truth table for each of these compound propositions.
    \begin{enumerate}
        \item $p \wedge \neg q$
        \item $p \vee \neg p$
        \item $(p \vee \neg q)\rightarrow q$
    \end{enumerate}
\end{frame}
\end{document}